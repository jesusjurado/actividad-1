\documentclass{article}
\usepackage[utf8]{inputenc}
\usepackage[spanish]{babel}
\usepackage{listings}
\usepackage{graphicx}
\graphicspath{ {images/} }
\usepackage{cite}

\begin{document}

\begin{titlepage}
    \begin{center}
        \vspace*{1cm}
            
        \Huge
        \textbf{Experimento Calistenia}
            
        \vspace{0.5cm}
        \LARGE
        Curso de Informatica 2 - actividad 1
            
        \vspace{1.5cm}
            
        \textbf{Jesus Daniel Jurado Hernandez}
            
        \vfill
            
        \vspace{0.8cm}
            
        \Large
        Departamento de Ingeniería Electrónica y Telecomunicaciones\\
        Universidad de Antioquia\\
        Medellín\\
        Marzo de 2021
            
    \end{center}
\end{titlepage}

\tableofcontents
\newpage
\section{Explicacion del ejercicio}\label{intro}
En este experimento le indicaremos a 3 personas que sigan al pie de la letra una serie de instrucciones. La solución al desafío es llevar unos objetos de una posición A a una posición B por medio de los pasos mostrados a continuacion.

\section{Instrucciones} \label{contenido}

Paso 1: Utilizando la mano derecha, se toma la hoja y se desplaza hacia un punto diferente, de forma que esta ya no este encima de las tarjetas.

\vspace{0.5cm}

paso 2: Con la mano derecha se levantan ambas tarjetas

\vspace{0.5cm}

paso 3: Con los dedos de su mano derecha agarramos las tarjetas unicamente por sus lados mas largos.

\vspace{0.5cm}
        
paso 4: Ayudandonse de la superficie de la mesa o de cualquier superficie plana, alinear perfectamente las tarjetas, permitiendo que estas se deslicen por sus dedos sin dejar de agarrarlas.

\vspace{0.5cm}

paso 5: Ubicar verticalmente las tarjetas encima del punto central de la hoja, sin dejar de agarrarlas con los dedos.

\vspace{0.5cm}

paso 6: Ubicar el dedo indice de la mano derecha en la mitad del lado superior de las tarjetas, sin dejar de agarrarlas por los costados con los otros dedos.

\vspace{0.5cm}

paso 7: Utilizando el dedo indice de nuestra mano derecha, hacemos presion hacia abajo a la tarjeta mas lejana a usted.

\vspace{0.5cm}

paso 8: Utilizando sus dedos pulgar y medio de su mano derecha, agarrar la tarjeta mas cercana a usted, todo esto mientras sigue ejerciendo presion en la otra tarjeta.

\vspace{0.5cm}

paso 9: Unir cuidadosamente los lados superiores de las tarjetas mientras separa lentamente los lados inferiores, todo esto ayudandose de sus dedos de apoyo de su mano derecha.

\vspace{0.5cm}

paso 10: Una vez se llega a un punto de equilibrio entre ambas tarjetas, soltar cuidadosamente el dedo indice de su mano derecha, y luego sus dedos pulgar y medio de la misma mano, procurando mantener el equilibrio de las tarjetas.

\vspace{0.5cm}

paso 11: si las tarjetas se cayeron, repita el proceso desde el paso 2




\end{document}
